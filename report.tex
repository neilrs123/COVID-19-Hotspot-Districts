\documentclass{article}
\usepackage{graphicx}
\usepackage{setspace}
\usepackage{hyperref}
\graphicspath{ {./images/} }
% if you need to pass options to natbib, use, e.g.:
%     \PassOptionsToPackage{numbers, compress}{natbib}
% before loading neurips_2019

% ready for submission
% \usepackage{neurips_2019}

% to compile a preprint version, e.g., for submission to arXiv, add add the
% [preprint] option:
%     \usepackage[preprint]{neurips_2019}

% to compile a camera-ready version, add the [final] option, e.g.:
\usepackage{caption}
\usepackage[preprint]{neurips_2019}
\usepackage{amsmath}
\DeclareMathOperator*{\argminA}{arg\,min} % Jan Hlavacek
\DeclareMathOperator*{\argminB}{argmin}   % Jan Hlavacek
\DeclareMathOperator*{\argminC}{\arg\min}   % rbp% thin space, limits underneath in displays
% to avoid loading the natbib package, add option nonatbib:
%     \usepackage[nonatbib]{neurips_2019}

\usepackage[utf8]{inputenc} % allow utf-8 input
\usepackage[T1]{fontenc}    % use 8-bit T1 fonts
\usepackage{hyperref}       % hyperlinks
\usepackage{url}            % simple URL typesetting
\usepackage{booktabs}       % professional-quality tables
\usepackage{amsfonts}       % blackboard math symbols
\usepackage{nicefrac}       % compact symbols for 1/2, etc.
\usepackage{microtype}      % microtypography

\title{CS685: Assignment 1}

% The \author macro works with any number of authors. There are two commands
% used to separate the names and addresses of multiple authors: \And and \AND.
%
% Using \And between authors leaves it to LaTeX to determine where to break the
% lines. Using \AND forces a line break at that point. So, if LaTeX puts 3 of 4
% authors names on the first line, and the last on the second line, try using
% \AND instead of \And before the third author name.

\author{
  Neil Rajiv Shirude \\
  Roll no:170429 \\
  \texttt{neilrs@iitk.ac.in} \\
 }
 

\begin{document}

\maketitle

\section{Solution to Part 10: Overall Conclusions}

\subsection{Choice of Covid-19 Data}

The folllowing datasets were available at https://api.covid19india.org/-
\begin{enumerate}
    \item Entire data: \url{https://api.covid19india.org/v4/data-all.json}
    \item Raw data: \url{https://api.covid19india.org/raw_data1.json}
    \item Data in CSV format: \url{https://api.covid19india.org/csv/latest/districts.csv} 
\end{enumerate}

I chose to use entire data (data-all.json), because of the following merits- 
\begin{itemize}
    \item Availability of data over entire period of analysis, unlike the data in CSV format (districts.csv)
    \item No repetition in the number of cases and no double counting, as in the case of raw data
    \item Well-documented 'confirmed' number of cases. (On the other hand, raw data has reliable 'hospitalized' data. But according to \href{https://www.who.int/indonesia/news/detail/08-03-2020-knowing-the-risk-for-covid-19#:~:text=Most\%20people\%20(about\%2080,are\%20at\%20greater\%20risk.}{WHO}, the hospitalization rate for COVID-patients is not uniform and is less than 20\%.
\end{itemize}

\subsection{Missing Values}

\subsubsection{Identification of Missing Values}

\begin{enumerate}
    \item For certain days, particularly in the beginning of the analysis period, district-level data was not available. Instead, state-level data was available.
    \item Certain states considered certain special COVID-19 cases as districts. These included-
    \begin{itemize}
        \item State Pool
        \item Foreign Evacuees
        \item Airport Quarantine
        \item Railway Quarantine
        \item BSF Camp
        \item Italians
        \item Other Districts
        \item Others, etc.
    \end{itemize}
\end{enumerate}

\subsubsection{Handling of Missing Values}

\begin{enumerate}
    \item For a certain day, if district-level data is not available and  only state-level data is available, I mapped the cases for a state to the district in the state with the highest population. 
    \item Over the period of analysis, there were approximately 38 lakh confirmed COVID-19 cases in India. The sum of the number of cases mapped to these absurd districts was less than 30,000 cases. As these absurd cases account for less than 1\% of the total casesover the analysis period, I ignored these cases (after consulting with sir during a discussion hour).
\end{enumerate}

\subsection{Overall Conclusions}

\subsubsection{Overall Results}

Over the entire period of analysis, the following districts were identified as the top-5 hotspots-
\begin{itemize}
    \item Neighborhood analysis hotspots-
    \begin{enumerate}
        \item Bengaluru urban
        \item Bhopal
        \item Delhi
        \item Surat
        \item Ahmedabad
    \end{enumerate}
    \item Neighborhood analysis coldspots-
    \begin{enumerate}
        \item Upper Dibang Valley
        \item Krishna
        \item South Tripura
        \item Kalimpong
        \item Unokoti
    \end{enumerate}
    \item State analysis hotspots-
    \begin{enumerate}
        \item Puducherry
        \item Bengaluru urban
        \item Chennai
        \item Raipur
        \item Patna
    \end{enumerate}
    \item State analysis coldspots-
    \begin{enumerate}
        \item Diu
        \item Krishna
        \item Lahaul and Spiti
        \item Vizianagaram
        \item Wayanad
    \end{enumerate}
\end{itemize}

\subsubsection{Interpretation of Overall Results}

According to this analysis, the districts with very high cases in comparison to their neighboring districts/ other districts in their state are identified as hotspot districts. If the spread of COVID-19 cases in these hotspot districts can be contained, the impact would be very high, since the disease has not spread in the vicinity of these districts.

Districts such as Pune and Mumbai that have reported the highest number of cases in India cannot be predominantly seen in these lists since their neighbors/ other districts in their state have also witnessed large scale spread of COVID-19.

Similarly, the districts having very low cases in comparison to their neighboring districts/ other districts in their state are identified as coldspot districts. These districts have done a better job at containing the spread of COVID-19 as compared to their neighbors.


\end{document}

